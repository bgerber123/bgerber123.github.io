\documentclass[12pt]{article}

\usepackage[utf8]{inputenc}
\usepackage[margin=1in,footskip=0.25in]{geometry}
\usepackage[english,activeacute]{babel}
\usepackage{amsmath}

\title{Bayesian  Model}
\author{Brian D. Gerber}

\begin{document}

\maketitle
\section{Bayesian Generalized Linear Model}
\subsection{Hierarchical Statistical Model with extra Variance Term for Unexplained Site Specific Variation}

\Large

We want to fit a model to our data, where the occurence of the species ($y_{i,j}$) for site $i$ in projected area $j$ is modeled as,

\begin{align*}
y_{i,j}\sim& \text{Bernoulli}(p_{i,j})\\
\text{logit}(p_{i,j}) &= \beta_{0} + \beta_{1}\times \text{dist.human}_{i,j}+ \epsilon_{i,j}\\
\epsilon_{i,j} \sim \text{Normal}(0,\sigma)
\end{align*}

\subsubsection{Priors}

\begin{align*}
\beta_{0} \sim& \text{Logistic}(0, 1)\\
\beta_{1} \sim& \text{Logistic}(0, 1)\\
\sigma \sim& \text{Uniform}(0,5)
\end{align*}


\pagebreak

\subsection{JAGS syntax for this model}

\begin{verbatim}
model {

# Priors
  b0 ~ dlogis(0,1)
  b1 ~ dlogis(0,1)
  sigma ~ dunif(0,5)
  tau <- 1/sigma^2

# Likelihood
  for (i in 1:N) {
     y[i] ~ dbern(p[i])
     logit(p[i]) <- b0 + b1*dist.human[i]+epsilon[i]
     epsilon[i]~ dnorm(0,tau)
  } #End loop
  
} #End Model
\end{verbatim}





\subsection{Hierarchical Statistical Model with Extra Variance Term for Unexplained Variation by Protected Area}

\Large

We want to fit a model to our data, where the occurence of the species ($y_{i,j}$) for site $i$ in projected area $j$ is modeled as,

\begin{align*}
y_{i,j}\sim& \text{Bernoulli}(p_{i,j})\\
\text{logit}(p_{i,j}) &= \beta_{0} + \beta_{1}\times \text{dist.human}_{i,j}+ \epsilon_{j}\\
\epsilon_{j} &\sim \text{Normal}(0,\sigma)
\end{align*}

\subsubsection{Priors}

\begin{align*}
\beta_{0} \sim& \text{Logistic}(0, 1)\\
\beta_{1} \sim& \text{Logistic}(0, 1)\\
\sigma \sim& \text{Uniform}(0,5)
\end{align*}


\pagebreak

\subsection{JAGS syntax for this model}

Note that the variable PA should be an index labled as 1,1,1,1....50 times for the first PA and then 2,2,2,2...50 times for the second PA and so on....

\begin{verbatim}
model {

# Priors
  b0 ~ dlogis(0,1)
  b1 ~ dlogis(0,1)
  sigma ~ dunif(0,5)
  tau <- 1/sigma^2

# Likelihood
  for (i in 1:N) {
     y[i] ~ dbern(p[i])
     logit(p[i]) <- b0 + b1*dist.human[i]+epsilon[PA[i]]
  } #End loop

#Random variation by PA
for(i in 1:N.PA){
  epsilon[j] ~ dnorm(0,tau)
 }
 
} #End Model
\end{verbatim}


\end{document}