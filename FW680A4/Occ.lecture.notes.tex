\documentclass[12pt]{article}

\usepackage[utf8]{inputenc}
\usepackage[english,activeacute]{babel}
 \usepackage{amsmath}
\usepackage{kbordermatrix}
\title{Occupancy Lecture}
\author{Brian D. Gerber}

\begin{document}

\maketitle

\section{Single Season Single-Species Occupancy}
\subsection{The Setup}

Assigned primate-occupancy paper.

\textbf{Questions}
\begin{itemize}
	\item why do we care where species are and are not?
	\item What are the data?
	\item why is this not just logistic regression?
	\item what is a season?
	\item What is the unit of replication?
\end{itemize}

Start with what we want.... we want to know occurence and the probabiliy a unit $i$ will be occupied

Discuss false-negatives and false positives

Show on board data for true occurence 0, 1, 1, etc.
Show on board data for detection data 0, 0, 1, etc.

What is the concern? Undersestimating occurence.

Problem also when using covariates

Drawn prob occupancy going up as a function of cover looking for cottontails?
Draw line for observed curve that goes down as cover increasing - implying what?


\subsection{The Process Model}

\textbf{Constant Occupancy}
\begin{align*}
z_{[i]} \sim \text{Bernoulli}(\psi)
\end{align*}


\noindent \textbf{Site-level Occupancy}
\begin{align*}
z_{[i]} \sim \text{Bernoulli}(\psi_{i})\\
\text{logit}(\psi_{i}) = \textbf{X}_{i}\boldsymbol{\beta}
\end{align*}


Next, we need to get our data to this process. What is out data?

\subsection{The detection model i}

\[y_{i}\sim
	\begin{cases}
	0\hspace{67pt}, \hspace{4pt}z_1=0\\
	\text{Binomial}(J_{i}, p_{i,j}) \hspace{1pt}, z_i=1, \hspace{3pt}
	\end{cases}\
\]


\subsection{The detection model i,j}

\begin{align*}
z_{[i]} \sim \text{Bernoulli}(\psi)
\end{align*}

Next, we need to get our data to this process. What is out data?

\[y_{i,j}\sim
	\begin{cases}
	0\hspace{60pt}, z_1=0\\
	\text{Bernoulli}(p_{i,j}) \hspace{2pt}, z_i=1, \hspace{2pt}
	\end{cases}\
\]

We could also write this as 

\begin{align*}
y_{i,j}\sim \text{Bernoullli}(p_{i,j} \times z_{i})
\end{align*}

Last, let's connect our detection model with some variables

\begin{align*}
\text{logit}(p_{i,j}) = \textbf{W}_{i,j}\boldsymbol{\alpha}
\end{align*}

\subsection{Priors}

\begin{align*}
\beta_{k} \sim \text{Logistic}(0,1)\\
\alpha_{k} \sim \text{Logistic}(0,1)\\
\end{align*}

\subsection{Additional Site-level Variance in Occurence}

\begin{align*}
\text{logit}(\psi_{i}) = \textbf{X}_{i}\boldsymbol{\beta}+ \sigma
\end{align*}

Need a new prior...

\begin{align*}
\sigma \sim \text{Uniform}(0, 4)
\end{align*}


\subsection{Go to Code}

\section{Multi-State Occupancy}

\textbf{Questions}
\begin{itemize}
	\item why multi-state?
\end{itemize}


\subsection{The States and Process Model}

The states of interest

\begin{align*}
\psi_1 = \text{Prob. Species Occurence with no Reproduction}\\
\psi_2 = \text{Prob. Species Occurence with Reproduction}\\
1 - \psi_1 - \psi_2 =  \text{Prob.  Not Occupied}\\
\psi^{*} =  \psi_1 + \psi_2 = \text{Species Occurence regardless of Reproduction}\\
\end{align*}


\begin{align*}
\Omega \sim
\begin{bmatrix}
           1- \psi_{1} - \psi_{2} \\
           \psi_1 \\
	 \psi_2
\end{bmatrix}
\end{align*}


\begin{align*}
z_{i} \sim \text{Categorical}(\Omega)
\end{align*}

Code - rmultinom(1,1,prob=c(0.2,0.5,0.1))


\subsection{Alt States}
\begin{align*}
\psi = \text{Prob. Species Occurence}\\
r = \text{Prob of Reproduction}\\
\end{align*}

\begin{align*}
\Omega \sim
\begin{bmatrix}
           1-\psi \\
           \psi\times(1-r) \\
	 \psi \times r
\end{bmatrix}
\end{align*}


\subsection{Detection Model}


\begin{align*}
y_{i,j} \sim \text{Categorical}(\textbf{P}_{\textbf{z}_{i},})
\end{align*}

dimensions to consider of this 
3x3 times 3x1 = 3 x1

\[
  \textbf{P} = \kbordermatrix{
    & \text{Not Seen} & \text{Seen without Repro} & \text{Seen with Repro} \\
   \text{Not Occ} & 1 & 0 & 0 \\
    \text{Occ without repro} & 1-p_2 & p_2 & 0\\
    \text{Occ with repro} & p_{3,1} &  p_{3,2}& p_{3,3}\\
  }
\]


\subsection{Priors}

\begin{align*}
\psi \sim \text{Unif}(0,1)\\
r \sim \text{Unif}(0,1)\\
p_{2} \sim \text{Unif}(0,1)\\
p_3 \sim \text{Dirichlet}(1,1,1)
\end{align*}


\end{document}
